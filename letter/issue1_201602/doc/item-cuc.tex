
\setcounter{section}{0}
\setcounter{figure}{0}
\graphicspath{{./figs/}}

\NewsTitle{Energy Optimization, Control and Markets Lab at the Catholic University of Chile}
\NewsAuthor{Matias Negrete-Pincetic, Daniel Olivares, Rodrigo Henriquez, and George Wenzel\\
Catholic University of Chile}

\begin{abstract}

The Energy Optimization, Control and Markets Lab (OCM-Lab) performs its activities at the Electrical Engineering Department at Pontificia Universidad Catolica de Chile. Led by two faculties, Assistant Professors Daniel Olivares and Matias Negrete Pincetic, focuses its activities on developing state of the art research on the boundaries of power systems, control, operation research and economics. Currently, the group has 15 postgraduate and undergraduate students and has international cooperation with several groups around the world including researchers from UC Berkeley, University of Notre Dame, University of Toronto, University of Waterloo and The University of Texas, Austin.
\end{abstract}


The electrical power system is a fundamental and vital part of a modern society's infrastructure: Its objective is to deliver reliable electrical energy to consumers efficiently. Many countries have decided to incorporate information technologies into the grid, in order to enhance the system reliability and efficiency and to harness more renewable energy sources, resulting in the smart grid vision. Along this transformation, the salient characteristics of electricity continue to impose stringent requirements on the power grid. Thus, it is challenging to maintain reliable operation, as well as functional markets, for this highly complex system. \\

Our research activities are in the realm of power and energy systems, combined with elements from decision and control sciences, economics and energy policy. On the past, we have been able to work on the understanding of the impact, viewed from the ideal competitive equilibrium setting, of dynamics, constraints, and uncertainty which are inherent to power and energy systems \cite{wannegkowshameysha11b}. In a general dynamic setting, we established many of the standard conclusions of the competitive equilibrium theory: Market equilibria are efficient, and average prices coincide with average marginal costs. However, these conclusions hold only on average and the dynamics of prices can be extreme. Price volatility, negative prices and price spikes hold even in a perfect competitive setting in which price manipulation is excluded. These findings along with other recent academic results and the empirical experience of the last decades illustrate the need for developing new models and tools for analyzing energy systems and its associated markets.\\

Developing such models and tools is a challenging task. Mainly, because energy markets require the coexistence of two coupled dynamical systems: a physical system driven by hard and soft engineering constraints in which reliability is one of the main objectives, and a market/financial system driven by self-interests of players in which economic efficiency is the leading metric. In the particular case of electricity markets and as a result of the many changes expected by the Smart Grid vision --- the use of information technologies, new energy policies, active demand participation and renewable energy sources among other ones --- increased levels of uncertainty and exotic dynamics will naturally emerge. These features will make even more challenging the operation and planning of energy systems and the coexistence with its associated markets.\\

Our research has been focused on understanding the interaction between these two complex systems, developing models and tools tailored for a grid with increased levels of uncertainty and dynamics, and proposing market designs and energy policies appropriate for this new setting. Some of our previous and current research activities include:\\

{\bf Electricity Market Design}: Development of alternative architectures for electricity markets based on the notion of multi-attribute products and the definition of contracts. Designing markets for flexibility. Investigate the role of the product definition in electricity auctions and its key role on market outcomes. Study of the design and outcomes of electricity auctions and capacity markets. Understanding the interaction of low marginal cost technologies in current market structures \cite{neg2015,negmey12,nayneg15}. \\

{\bf Impact of Dynamics and Uncertainty}: Study of the impact of volatility and dynamics on electricity markets. Using current operational schemes and typical assumptions, it was showed that under current electricity market designs the volatility of wind can reduce its inherent value. Moreover, the loss in social welfare is skewed to the supplier. The results also illustrate the need to move beyond snap-shot-based models for analyzing electricity markets and reinforce the idea of multi-attribute products \cite{meynegwankowsha10}.\\

{\bf Resource Planning Models}: Development of models and tools for planning energy systems with increased levels of uncertainty and dynamics. The inherent complexity, uncertainty and dynamics is handled by using techniques and methodologies from decision and control, and simulation and learning. Discussion about the need to upgrade usual power and energy reliability metrics. Study of composite reliability metrics suitable for planning studies in a market environment. Upgrading of SWITCH planning model.\\

{\bf Demand Response Aggregators}:  Our research group focus in developing new operation and economics models for Demand Side Management (DSM), for example a Demand Response (DR) aggregator that participates in electricity markets, and study the value of being flexible for the end-user. As mentioned, it is well known that DSM can give several benefits for the power systems, but what incentives can be used and how to promote the participation of the consumers is a challenging task that our group is trying to understand.\\

{\bf Tools and Models for Vehicle-to-Grid}: Electric vehicles are expected to have a key role on the grid of the future. In this realm our research have been focusing on control and market aspects of the vehicle to grid concept. On the control side, we have been developing algorithms for the real-time control operation of a fleet of electric vehicles participating on ancillary services markets \cite{juul15}. On the market side, our work have been focusing on pricing schemes for EVs charging stations \cite{liu15}. \\

{\bf Urban Microgrids and Power System Resilience}: Distribution networks are especially sensitive to natural disasters as they are normally designed with little redundancy and little autonomous control; thus, the loss of a line or a transformer may cause a disruption of the supply to an area for a long period of time until the damaged asset can be repaired. This research aims to overcome this weakness by using urban microgrids. Microgrids are distribution-scale networks with clusters of loads, distributed generation and energy storage systems that operate autonomously to reliably supply electricity. A key aspect of micro grids is their ability to seamlessly switch between operating with or without a connection to a wider power system. This feature makes microgrids a potential solution to reduce the number of power outages following a natural disaster \cite{oli14,oli15,oli15b}.\\

\bibliographystyle{plain}
\begin{thebibliography}{10}

\bibitem{wannegkowshameysha11b}
G.~Wang, M.~Negrete-Pincetic, A.~Kowli, E.~Shafieepoorfard, S.~Meyn, and
  U.~Shanbhag, ``Dynamic competitive equilibria in electricity markets,'' in
  {\em Control and Optimization Theory for Electric Smart Grids}
  (A.~Chakrabortty and M.~Illic, eds.), Springer, 2011.

\bibitem{neg2015}
M.~Negrete-Pincetic, L.~de~Castro, and H.~A. Pulgar-Painemal, ``Electricity
  supply auctions: Understanding the consequences of the product definition,''
  {\em International Journal of Electrical Power \& Energy Systems}, vol.~64,
  pp.~285 -- 292, 2015.

\bibitem{negmey12}
M.~Negrete-Pincetic and S.~Meyn, ``{Markets for Differentiated Electric Power
  Products in a Smart Grid Environment},'' in {\em IEEE PES 12: Power Energy
  Society General Meeting}, 2012.

\bibitem{nayneg15}
A.~Nayyar, M.~Negrete-Pincetic, K.~Poolla, and P.~Varaiya,
  ``Duration-differentiated energy services with a continuum of loads,'' {\em
  Control of Network Systems, IEEE Transactions on}, vol.~PP, no.~99, pp.~1--1,
  2015.

\bibitem{meynegwankowsha10}
S.~Meyn, M.~Negrete-Pincetic, G.~Wang, A.~Kowli, and E.~Shafieepoorfard, ``The
  value of volatile resources in electricity markets,'' in {\em Proc. of the
  49th IEEE Conf. on Dec. and Control}, pp.~1029 --1036, 2010.

\bibitem{juul15}
F.~Juul, M.~Negrete-Pincetic, J.~MacDonald, and D.~Callaway, ``Real-time
  scheduling of electric vehicles for ancillary services,'' in {\em Power
  Energy Society General Meeting, 2015 IEEE}, pp.~1--5, July 2015.

\bibitem{liu15}
J.~Liu, M.~Negrete-Pincetic, and V.~Gupta, ``Optimal charging profiles and
  pricing strategies for electric vehicle charging stations,'' in {\em
  PowerTech, 2015 IEEE Eindhoven}, pp.~1--6, June 2015.

\bibitem{oli14}
D.~Olivares, A.~Mehrizi-Sani, A.~Etemadi, C.~Canizares, R.~Iravani,
  M.~Kazerani, A.~Hajimiragha, O.~Gomis-Bellmunt, M.~Saeedifard,
  R.~Palma-Behnke, G.~Jimenez-Estevez, and N.~Hatziargyriou, ``Trends in
  microgrid control,'' {\em Smart Grid, IEEE Transactions on}, vol.~5,
  pp.~1905--1919, July 2014.

\bibitem{oli15}
D.~Saez, F.~Avila, D.~Olivares, C.~Canizares, and L.~Marin, ``Fuzzy prediction
  interval models for forecasting renewable resources and loads in
  microgrids,'' {\em Smart Grid, IEEE Transactions on}, vol.~6, pp.~548--556,
  March 2015.

\bibitem{oli15b}
D.~Olivares, J.~Lara, C.~Canizares, and M.~Kazerani, ``Stochastic-predictive
  energy management system for isolated microgrids,'' {\em Smart Grid, IEEE
  Transactions on}, vol.~6, pp.~2681--2693, Nov 2015.

\end{thebibliography}


